\usepackage{etex}	% avoid memory errors

\newenvironment{dummy}{}{}	% for saving LaTeX environments

% crossing out stuff, Option für normale Darstellung der \emph{} tags (sonst unterstrichen)
\usepackage[normalem]{ulem}

\usepackage{tipa}
\usepackage{leipzig}
\usepackage{expex}
%\lingset{everygla= , aboveglftskip=-15pt, aboveexskip=.5ex, belowexskip=.5ex, interpartskip=0.3ex plus .2ex minus .2ex, belowpreambleskip=0.2ex plus .2ex minus .2ex}	% don't italicise first line of examples
	%,
%\lingset{everygla= , aboveglftskip=0\baselineskip}	% don't italicise first line of examples
\lingset{glhangstyle=none, everygla= , everyglb=\small, aboveglftskip=0pt, aboveexskip=0.2\baselineskip, belowexskip=0.2\baselineskip, labeloffset=3pt, textoffset=8pt, numoffset=0pt, interpartskip=0.3ex plus .2ex minus .2ex, belowpreambleskip=0.2ex plus .2ex minus .2ex}


\resetcountonoverlays{excnt}		% avoid upcounting example numbers on overlay slides

%%%%%%%%%%%%%%%%%%%%%%%
%%% TikZ-related stuff 
%%%%%%%%%%%%%%%%%%%%%%%

\usepackage{tikz}
\usetikzlibrary{arrows,arrows.meta,shapes.misc,shadows,shapes.multipart,positioning,decorations.markings}
%\usepackage{qtree}
\usepackage{tikz-qtree}
\usepackage{tikz-qtree-compat}
\tikzset{every tree node/.style={align=center, anchor=north}}	% allows multiple lines in tree node (as side effect, cf. handbook)

\usepackage{tikz-Nodeconnect}


% at some point to implement: (optional) argument positioning of label (north, south)
% first argument label of main node
% second argument text of main node
% third argument tag
\newcommand{\inlineLabel}[3]{%
    \begin{tikzpicture}[baseline=(#1.base), txt/.style={shape=rectangle, inner sep=0pt}, remember picture]% the baseline key ensures that nodes won't shift up if there's text with descenders, and the txt style removes extra spacing so you can use this inline
    \node[txt] (#1) {#2};% the third argument is the contents of the main node
    \node[above] at (#1.north) {\footnotesize{#3}};% the second argument is the tag; you can play with the positioning as necessary
    \end{tikzpicture}%
    }
    

\newcommand{\anchornode}[2]{%
    \begin{tikzpicture}[baseline=(#1.base), txt/.style={shape=rectangle, inner sep=0pt}, remember picture]% the baseline key ensures that nodes won't shift up if there's text with descenders, and the txt style removes extra spacing so you can use this inline
    \node[txt] (#1) {#2};% the third argument is the contents of the main node
%   \node[above] at (#1.north) {\footnotesize{#3}};% the second argument is the tag; you can play with the positioning as necessary
    \end{tikzpicture}%
    }


%strike through arrow
\newcommand*{\StrikeThruDistance}{0.17cm}%
\newcommand*{\StrikeThru}{\StrikeThruDistance,\StrikeThruDistance}%


\tikzset{strike thru arrow/.style={
    decoration={markings, mark=at position 0.5 with {
        \draw [red, thick,-] 
            ++ (-\StrikeThruDistance,-\StrikeThruDistance) 
            -- ( \StrikeThruDistance, \StrikeThruDistance);
	\draw [red, thick,-] 
            ++ (-\StrikeThruDistance, \StrikeThruDistance) 
            -- ( \StrikeThruDistance, -\StrikeThruDistance);}
    },
    postaction={decorate},
}}

% end TikZ
%%%%%%%%%%%%%%%%%%%%%%%%%%%%%%%%%%%%%%%%

\newcommand{\unt}[1]{\hbox{}$_\text{#1}$}

\usepackage{xcolor}

\definecolor{light-gray}{gray}{.9}
\definecolor{mid-gray}{gray}{.8}
\definecolor{dark-gray}{gray}{.5}

%%%%%%%%%%%%%%%%%%
%% LAYOUT
%%%%%%%%%%%%%%%%%%%%%%


%\setlength{\leftmargini}{3pt}
%\setlength{\leftmarginii}{5pt}

% test 
\definecolor{darkgreen}{RGB}{0,100,0}
\definecolor{lightergreen}{RGB}{34,134,34}
\definecolor{lightgreen}{RGB}{151, 237, 166}
\definecolor{alertgreen}{RGB}{11,168,37}

%\setbeamercolor{title separator}{fg=darkgreen , bg=lightgreen}
%\setbeamercolor{progress bar in section page}{fg=darkgreen , bg=lightgreen}
%\setbeamercolor{alerted text}{fg=alertgreen}

\usepackage[autostyle]{csquotes}	% for properly typeset quotes
